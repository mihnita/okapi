\documentclass{article}
 
%Russian-specific packages
%--------------------------------------
\usepackage[T2A]{fontenc}
\usepackage[utf8]{inputenc}
\usepackage[russian]{babel}
%--------------------------------------

%Hyphenation rules
%--------------------------------------
\usepackage{hyphenat}
\hyphenation{ма-те-ма-ти-ка вос-ста-нав-ли-вать}
%--------------------------------------

\begin{document}

\tableofcontents

\begin{abstract}
  Это вводный абзац в начале документа.
\end{abstract}

\section{Предисловие}
 Этот текст будет на русском языке. Это демонстрация того, что символы кириллицы
 в сгенерированном документе (Compile to PDF) отображаются правильно.
 Для этого Вы должны установить нужный  язык (russian) 
и необходимую кодировку шрифта (T2A).

\section{Математические формулы}
Кириллические символы также могут быть использованы в математическом режиме.

\begin{equation}
  S_\textup{ис} = S_{123}
\end{equation}

\end{document}